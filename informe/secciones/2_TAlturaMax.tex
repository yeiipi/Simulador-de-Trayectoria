\section{Tiempos para Alcanzar la Altura Máxima}
\label{sec:tiempor_para_alcanzar_la_altura_maxima}

El punto en el que se obtienen la altura máxima en un lanzamiento parabólico,
es cuando la velocidad en y \(v_y\) toma valor  \(0\), por lo tanto teniendo la
ecuación de \(v_y\) se puede calcular el tiempo que un objeto toma en llegar a
tal estado, luego:

\begin{align*}
    v_y &=  v_0\sin\theta -gt \\
    0 &=  v_0\sin\theta -gt \\
    t &= \frac{v_0\sin\theta}{g} \\
\end{align*}

Teniendo eso en cuenta y sabiendo que la gravedad se toma como una constante
(no cambia), es necesario analizar en lo que está aportando al crecimiento de
esta función la velocidad inicial \(v_0\) y el ángulo de lanzamiento
\(\theta\). Como al realizar un experimento la velocidad inicial será definida
y por lo tanto constante se puede decir que el incremento de tiempo es
proporcional a la velocidad inicial, lo más interesante se encuentra en
\(\sin\theta\) ya que \(\sin\) es una función cíclica y con un rango
\(\left[-1, 1\right] \), esto generando un efecto de partición en lo que la
velocidad puede aportar a la función, es decir si el el ángulo es \(\theta =0\)
entonces  \(t=0\) y si el ángulo es \(\theta = \frac{\pi}{2}\) entonces
\(t=\frac{v_0}{g}\) asegurando el mayor tiempo hasta
llegar a la altura máxima.
