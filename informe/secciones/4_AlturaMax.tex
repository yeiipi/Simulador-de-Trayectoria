\section{Altura Máxima}
\label{sec:altura_maxima}


Para calcular la altura máxima podemos aprovechar aquellas ecuación que ya
hemos obtenido. En este caso utilizaremos el tiempo que toma el proyectil en
llegar a la altura máxima:

\begin{align*}
    y &=  v_{0}\sin\theta t - gt \\
      &= \frac{v_{0}^{2}\sin^{2}\theta}{g}-\frac{v_{0}^{2}\sin^{2}\theta}{2g} \\
      &= \frac{v_{0}^{2}\sin^{2}\theta}{2g} \\
\end{align*}

Al igual que las dos propiedades anteriores, esta función tambien tiene
\(\sin\) en el númerador determinado la proporción del crecimiento de la
función por lo tanto la altura máxima se encuentra cuando \(\theta =
\frac{\pi}{2}\).






















