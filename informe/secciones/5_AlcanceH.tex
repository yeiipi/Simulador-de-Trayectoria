\section{Alcance Horizontal}
\label{sec:alcance_horizontal}


 Para calcular esta instancia uno puede apoyarse de las propiedades que ya se
 han mencionado bajo la ecuación de la distancia horizontal:

 \begin{align*}
     x &=  v_{0x}t \\
     &= v_{0}\cos\theta \frac{2v_{0}\sin\theta}{g}\\
     &= \frac{2v_{0}^{2}\cos\theta \sin\theta}{g} \\
     &= \frac{v_{0}^{2}\left(2\cos\theta \sin\theta\right) }{g} \\
 \end{align*}

 Ahora teniendo en cuenta la identidad trigonométrica \(2\cos\theta \sin\theta
 = \sin 2\theta \) \cite{Weisstein}, la distancia máxima se puede dejar como:

 \begin{align*}
     x &= \frac{v_{0}^{2}\sin 2\theta }{g} \\
 \end{align*}

 En esté caso acontece una pequeña diferencia con respecto a las funciones
 encontradas anteriormente,  pues ahora \(\sin\) no fluctua con respecto a
 \(\theta\) sino con respecto a \(2\theta\), por lo tanto sea \(\alpha =
 2\theta\) entonces \(\sin\alpha = 1\) cuando \(\alpha  = \frac{\pi}{2}\) por
 lo tanto el alcance horizontal máximo se obtien cuando \(\theta =
 \frac{\pi}{4}\).



























