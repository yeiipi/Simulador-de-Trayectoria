\section{Tiempo de Vuelo}
\label{sec:tiempo_de_vuelo}

El tiempo de vuelo es un factor que se decide cuando el proyectil ya ha
retornado al suelo, por lo tanto \(y=0\), así si utilizamos la ecuación de
distancia en  \(y\), es posible encontrar el tiempo que toma en retornar al
suelo:

\begin{align*}
    y &=  v_{0y}t-\frac{1}{2}gt^{2} \\
    0 &=  v_{0y}t-\frac{1}{2}gt^{2} \\
    0 &=  v_{0y}-\frac{1}{2}gt \\
    t &=  \frac{2v_{0y}}{g} \\
    t &=  \frac{2v_0\sin\theta }{g} \\
\end{align*}

Como la función también está en términos de \(\sin\) se puede utilizar la misma
propiedad, garantizando que el mayor tiempo de vuelo se obtiene cuando \(\theta
= \frac{\pi }{2}\).
